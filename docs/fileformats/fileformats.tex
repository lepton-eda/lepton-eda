% gEDA - GPL Electronic Design Automation
% fileformats.tex - gEDA/gaf File Formats 
% Copyright (C) 2002 Ales V. Hvezda
%
% This program is free software; you can redistribute it and/or modify
% it under the terms of the GNU General Public License as published by
% the Free Software Foundation; either version 2 of the License, or
% (at your option) any later version.
%
% This program is distributed in the hope that it will be useful,
% but WITHOUT ANY WARRANTY; without even the implied warranty of
% MERCHANTABILITY or FITNESS FOR A PARTICULAR PURPOSE.  See the
% GNU General Public License for more details.
%
% You should have received a copy of the GNU General Public License
% along with this program; if not, write to the Free Software
% Foundation, Inc., 675 Mass Ave, Cambridge, MA 02139, USA.

\documentclass{article}
\usepackage{epsfig}
% This line will enable hyperlinks in the PDF output
% file.
\usepackage[ps2pdf,breaklinks=true,colorlinks=true]{hyperref}

\setlength{\parindent}{0pt}
\setlength{\parskip}{1ex plus 0.5ex minus 0.2ex}

\title{gEDA/gaf File Format Document}
\author{Ales V. Hvezda, ahvezda@geda.seul.org\\
        \\
        This document is released under GFDL\\
	(\url{http://www.gnu.org/copyleft/fdl.html})}
\date{October 4th, 2003}

\begin{document}

\maketitle
\newpage

\tableofcontents
\newpage


\section{Overview}

This file is the official documentation for the file formats in gEDA/gaf
(Gschem And Friends).  The primary file format used in gEDA/gaf is the
schematic/symbol format.  Files which end with .sch or .sym are schematics
or symbol files.  Until there is another file type in gEDA/gaf, then this
document will only cover the symbol/schematic file format.  

This file format document is current as of gEDA/gaf version 20031004.  This
document covers file format version 1.


\section{Coordinate Space}
\begin{itemize}
 \item All coordinates are in mils (1/1000 or an inch).  This is an arbitrary decision. Remember in there is no concept of physical lengths/dimensions in schematics
and symbols (for schematic capture only).  
 \item Origin is in lower left hand corner.
 \item The size of the coordinate space is unlimited, but it is recommended that all objects stay within (120.0, 90.0) (x, y inches).
 \item It is generally advisable to have positive x and y coordinates, however, negative coordinates work too, but not recommended.
\end{itemize}

The following figure shows how the coordinate space is setup:

\begin{center}
\epsfbox{coords.eps}
\end{center}

X axis increases going to the right.  Y axis increase going up.
Coordinate system is landscape and corresponds to a sheet of paper turned
on its side.


\section{Filenames}

Symbols end in .sym.  The only symbol filename convention that is used in
gEDA/gaf is that if there are multiple instances of a symbol with the
same name (like a 7400), then a -1, -2, -3, ... -N suffix is added to
the end of the filename.  Example: 7400-1.sym, 7400-2.sym, 7400-3.sym...

Schematics end in .sch.  There used to be a schematic filename convention
(adding a -1 .. -N to the end of the basename), but this convention is now
obsolete.  Schematic filenames can be anything that makes sense to the 
creator.

\section{Object types}

A schematic/symbol file for gEDA/gaf consists of:

\begin{itemize}
 \item A version (v) as the first item in the file.  This is required.
 \item Any number of objects and the correct data.  Objects are specified 
       by an "object type"
 \item Most objects are a single line, however text objects are two lines long.
 \item No blank lines at the end of the file (these are ignored by the tools)
 \item For all enumerated types in the gEDA/gaf file formats, the field takes
       on the numeric value.
\end{itemize}

The "object type" id is a single letter and this id {\bf must} start in the 
first column.  The object type id is case sensitive.

The schematic and symbol files share the same file layout.  A symbol
is nothing more than a collection of primitive objects (lines, boxes,
circles, arcs, text, and pins).  A schematic is a collection of symbols
(components), nets, and buses.

The following sections describe the specifics of each recognized object 
type.  Each section has the name of the object, which file type (sch/sym)
the object can appear in, the format of the data, a description of
each individual field, details and caveats of the fields, and finally
an example with description.

For information on the color index (which is used in practically all objects),
see the Color section.

\subsection{version}

Valid in: Schematic and Symbol files

{\bf type version}

\begin{table}[h]
\begin{tabular}{|l|l|l|} \hline
Field 		& Type/unit 	& Description \\ \hline 
\hline
{\bf type} 	& char 		& v \\ \hline
{\bf version} 	& int 		& version of gEDA/gaf that wrote this file \\ \hline
{\bf fileformat\_version} 	& int 		& gEDA/gaf file format version number \\ \hline
\end{tabular}
\end{table}

\begin{itemize}
\item The type is a lower case "v" (as in Victor).
\item This object must be in every file used or created by the gEDA/gaf tools.
\item The format of the first version field is YYYYMMDD.  
\item The version number is not an arbitrary timestamp.  Do not make up a 
      version number and expect the tools to behave properly.  
\item The "version of gEDA/gaf that wrote this file" was used in all versions of gEDA/gaf up to 20030921 as the file formats version.  This field should no longer be used to determine the file format.  It is used for information purposes only now. 
\item Starting at and after gEDA/gaf version 20031004, the fileformat\_version field is used to determine the file format version.  All file format code should key off of this field.
\item fileformat\_version increases when the file format changes.
\item The starting point for fileformat\_version is 1.
\item fileformat\_version is just an integer with no minor number.
\end{itemize}

Valid versions include: \newline
19990601, 19990610, 19990705, 19990829, 19990919, 19991011, 20000220, 20000704,
20001006, 20001217, 20010304, 20010708, 20010722, 20020209, 20020414, 20020527,
20020825, 20021103, 20030223, 20030525, 20030901, 20030921, 20031004

Keep in mind that each of the listed versions might have had file format
variations.  This document only covers the last version's file format.

Example: \newline
{\tt v 20021004 1}


\subsection{line}

Valid in: Schematic and Symbol files

{\bf type x1 y1 x2 y2 color width capstyle dashstyle dashlength dashspace}

\begin{table}[h]
\begin{tabular}{|l|l|l|} \hline
Field 		& Type/unit 	& Description \\ \hline 
\hline
{\bf type} 	& char 		& L \\ \hline
{\bf x1} 	& int/mils 	& First X coordinate \\ \hline
{\bf y1} 	& int/mils 	& First Y coordinate \\ \hline
{\bf x2} 	& int/mils 	& Second X coordinate \\ \hline
{\bf y2} 	& int/mils 	& Second Y coordinate \\ \hline
{\bf color} 	& int 		& Color index \\ \hline
{\bf width} 	& int/mils 	& Width of line \\ \hline
{\bf capstyle} 	& int 		& Line cap style \\ \hline
{\bf dashstyle} & int 		& Type of dash style \\ \hline
{\bf dashlength}& int 		& Length of dash \\ \hline
{\bf dashspace} & int 		& Space inbetween dashes \\ \hline
\end{tabular}
\end{table}

\begin{itemize}
\item The capstyle is an enumerated type: 
\begin{itemize}
	\item END\_NONE = 0
	\item END\_SQUARE = 1
	\item END\_ROUND = 2
\end{itemize}
\item The dashstyle is an enumerated type: 
\begin{itemize}
	\item TYPE\_SOLID = 0 
	\item TYPE\_DOTTED = 1
	\item TYPE\_DASHED = 2
	\item TYPE\_CENTER = 3
        \item TYPE\_PHANTOM = 4
\end{itemize}
\item The dashlength parameter is not used for TYPE\_SOLID and TYPE\_DOTTED.  
      This parameter should take on a value of -1 in these cases.
\item The dashspace paramater is not used for TYPE\_SOLID.
      This parameter should take on a value of -1 in these case.
\end{itemize}

Example:\newline
{\tt L 23000 69000 28000 69000 3 40 0 1 -1 75}

A line segment from (23000, 69000) to (28000, 69000) with color index 3, 
40 mils thick, no cap, dotted line style, and with a spacing of 75 mils 
in between each dot.


\subsection{box}

Valid in: Schematic and Symbol files

{\bf type x y width height color width capstyle dashtype dashlength dashspace filltype fillwidth angle1 pitch1 angle2 pitch2 }

\begin{table}[h]
\begin{tabular}{|l|l|l|} \hline
Field 		& Type/unit 	& Description \\ \hline 
\hline
{\bf type} 	& char		& B \\ \hline
{\bf x} 	& int/mils	& Lower left hand X coordinate \\ \hline 
{\bf y} 	& int/mils	& Lower left hand Y coordinate \\ \hline
{\bf width} 	& int/mils 	& Width of the box (x direction) \\ \hline
{\bf height} 	& int/mils 	& Height of the box (y direction) \\ \hline
{\bf color} 	& int 		& Color index \\ \hline
{\bf width} 	& int/mils 	& Width of lines \\ \hline
{\bf capstyle} 	& int/mils	& Line cap style \\ \hline
{\bf dashstyle} & int 		& Type of dash style \\ \hline
{\bf dashlength}& int/mils	& Length of dash \\ \hline
{\bf dashspace} & int/mils	& Space inbetween dashes \\ \hline
{\bf filltype}  & int		& Type of fill \\ \hline
{\bf fillwidth} & int/mils	& Width of the fill lines \\ \hline
{\bf angle1} 	& int/degrees	& First angle of fill \\ \hline
{\bf pitch1} 	& int/mils	& First pitch/spacing of fill \\ \hline
{\bf angle2} 	& int/degrees 	& Second angle of fill \\ \hline
{\bf pitch2} 	& int/mils	& Second pitch/spacing of fill \\ \hline
\end{tabular}
\end{table}

\begin{itemize}
\item The capstyle is an enumerated type: 
\begin{itemize}
	\item END\_NONE = 0
	\item END\_SQUARE = 1
	\item END\_ROUND = 2
\end{itemize}
\item The dashstyle is an enumerated type: 
\begin{itemize}
	\item TYPE\_SOLID = 0 
	\item TYPE\_DOTTED = 1
	\item TYPE\_DASHED = 2
	\item TYPE\_CENTER = 3
        \item TYPE\_PHANTOM = 4
\end{itemize}
\item The dashlength parameter is not used for TYPE\_SOLID and TYPE\_DOTTED.  
      This parameter should take on a value of -1 in these cases.
\item The dashspace paramater is not used for TYPE\_SOLID.
      This parameter should take on a value of -1 in these case.
\item The filltype parameter is an enumerated type: 
\begin{itemize}
	\item FILLING\_HOLLOW = 0
	\item FILLING\_FILL = 1
	\item FILLING\_MESH = 2 
	\item FILLING\_HATCH = 3
        \item FILLING\_VOID = 4  {\bf unused}
\end{itemize}
\item If the filltype is 0 (FILLING\_HOLLOW), then all the fill parameters 
      should take on a value of -1.
\item The fill type FILLING\_FILL is a solid color fill.
\item The two pairs of pitch and spacing control the fill or hatch if the
      fill type is FILLING\_MESH. 
\item Only the first pair of pitch and spacing are used if the fill type is
      FILLING\_HATCH.
\end{itemize}

Example:\newline
{\tt B 33000 67300 2000 2000 3 60 0 2 75 50 0 -1 -1 -1 -1 -1}

A box with the lower left hand corner at (33000, 67300) and a width and height
of (2000, 2000), color index 3, line width of 60 mils, no cap, dashed line 
type, dash length of 75 mils, dash spacing of 50 mils, no fill, rest parameters 
unset.

\subsection{circle}

Valid in: Schematic and Symbol files

{\bf type x y radius color width capstyle dashtype dashlength dashspace filltype fillwidth angle1 pitch1 angle2 pitch2 }

\begin{table}[h]
\begin{tabular}{|l|l|l|} \hline
Field 		& Type/unit 	& Description \\ \hline 
\hline
{\bf type} 	& char		& V \\ \hline
{\bf x} 	& int/mils 	& Center X coordinate \\ \hline 
{\bf y} 	& int/mils	& Center Y coordinate \\ \hline
{\bf radius} 	& int/mils	& Radius of the circle \\ \hline
{\bf color} 	& int		& Color index \\ \hline
{\bf width} 	& int/mils	& Width of circle line \\ \hline
{\bf capstyle} 	& int/mils  	& 0 {\bf unused} \\ \hline
{\bf dashstyle} & int		& Type of dash style \\ \hline
{\bf dashlength}& int/mils	& Length of dash \\ \hline
{\bf dashspace} & int/mils	& Space inbetween dashes \\ \hline
{\bf filltype} 	& int		& Type of fill \\ \hline
{\bf fillwidth} & int/mils	& Width of the fill lines \\ \hline
{\bf angle1} 	& int/degrees	& First angle of fill \\ \hline
{\bf pitch1} 	& int/mils	& First pitch/spacing of fill \\ \hline
{\bf angle2} 	& int/degrees	& Second angle of fill \\ \hline
{\bf pitch2} 	& int/mils	& Second pitch/spacing of fill \\ \hline
\end{tabular}
\end{table}

\begin{itemize}
\item The dashstyle is an enumerated type: 
\begin{itemize}
	\item TYPE\_SOLID = 0 
	\item TYPE\_DOTTED = 1
	\item TYPE\_DASHED = 2
	\item TYPE\_CENTER = 3
        \item TYPE\_PHANTOM = 4
\end{itemize}
\item The dashlength parameter is not used for TYPE\_SOLID and TYPE\_DOTTED.  
      This parameter should take on a value of -1 in these cases.
\item The dashspace paramater is not used for TYPE\_SOLID.
      This parameter should take on a value of -1 in these case.
\item The filltype parameter is an enumerated type: 
\begin{itemize}
	\item FILLING\_HOLLOW = 0
	\item FILLING\_FILL = 1
	\item FILLING\_MESH = 2 
	\item FILLING\_HATCH = 3
        \item FILLING\_VOID = 4  {\bf unused}
\end{itemize}
\item If the filltype is 0 (FILLING\_HOLLOW), then all the fill parameters 
      should take on a value of -1.
\item The fill type FILLING\_FILL is a solid color fill.
\item The two pairs of pitch and spacing control the fill or hatch if the
      fill type is FILLING\_MESH. 
\item Only the first pair of pitch and spacing are used if the fill type is
      FILLING\_HATCH.
\end{itemize}

Example:\newline
{\tt V 38000 67000 900 3 0 0 2 75 50 2 10 20 30 90 50}

A circle with the center at (38000, 67000) and a radius of 900 mils, color 
index 3, line width of 0 mils (smallest size), no cap, dashed line 
type, dash length of 75 mils, dash spacing of 50 mils, mesh fill, 10 mils
thick mesh lines, first mesh line: 20 degrees, with a spacing of 30 mils, 
second mesh line: 90 degrees, with a spacing of 50 mils.


\subsection{arc}

Valid in: Schematic and Symbol files

{\bf type x y radius startangle sweepangle color width capstyle dashtype dashlength dashspace }

\begin{table}[h]
\begin{tabular}{|l|l|l|} \hline
Field 		& Type/unit 	& Description \\ \hline 
\hline
{\bf type} 	& char 		& A \\ \hline
{\bf x} 	& int/mils	& Center X coordinate \\ \hline 
{\bf y} 	& int/mils	& Center Y coordinate \\ \hline
{\bf radius} 	& int/mils	& Radius of the arc \\ \hline
{\bf startangle}& int/degrees 	& Starting angle of the arc \\ \hline
{\bf sweepangle}& int/degrees	& Amount the arc sweeps \\ \hline
{\bf color} 	& int		& Color index \\ \hline
{\bf width} 	& int/mils	& Width of circle line \\ \hline
{\bf capstyle} 	& int		& Cap style \\ \hline
{\bf dashstyle} & int		& Type of dash style \\ \hline
{\bf dashlength}& int/mils	& Length of dash \\ \hline
{\bf dashspace} & int/mils	& Space inbetween dashes \\ \hline
\end{tabular}
\end{table}

\begin{itemize}
\item The startangle can be negative, but not recommended.
\item The sweepangle can be negative, but not recommended.
\item The capstyle is an enumerated type: 
\begin{itemize}
	\item END\_NONE = 0
	\item END\_SQUARE = 1
	\item END\_ROUND = 2
\end{itemize}
\item The dashstyle is an enumerated type: 
\begin{itemize}
	\item TYPE\_SOLID = 0 
	\item TYPE\_DOTTED = 1
	\item TYPE\_DASHED = 2
	\item TYPE\_CENTER = 3
        \item TYPE\_PHANTOM = 4
\end{itemize}
\item The dashlength parameter is not used for TYPE\_SOLID and TYPE\_DOTTED.  
      This parameter should take on a value of -1 in these cases.
\item The dashspace paramater is not used for TYPE\_SOLID.
      This parameter should take on a value of -1 in these case.
\end{itemize}

Example:\newline
{\tt A 30600 75000 2000 0 45 3 0 0 3 75 50}

An arc with the center at (30600, 75000) and a radius of 2000 mils, a 
starting angle of 0, sweeping 45 degrees, color index 3, line width of 0 mils 
(smallest size), no cap, center line type, dash length of 75 mils, dash 
spacing of 50 mils.


\subsection{text}

Valid in: Schematic and Symbol files

{\bf type x y color size visibility show\_name\_value angle alignment}\newline
{\bf string}

\begin{table}[h]
\begin{tabular}{|l|l|l|} \hline
Field 		& Type/unit 	& Description \\ \hline 
\hline
{\bf type} 	& char 		& T \\ \hline
{\bf x} 	& int/mils	& First X coordinate \\ \hline
{\bf y} 	& int/mils	& First Y coordinate \\ \hline
{\bf color} 	& int		& Color index \\ \hline
{\bf size} 	& int/points	& Size of text \\ \hline
{\bf visibility}& int		& Visibility of text \\ \hline
{\bf show\_name\_value} & int	& Attribute visibility control \\ \hline
{\bf angle} 	& int/degrees	& Angle of the text \\ \hline
{\bf alignment} & int		& Alignment/origin of the text \\ \hline
{\bf string} 	& string	& The text string, on a seperate line \\ \hline
\end{tabular}
\end{table}

\begin{itemize}
\item This object is a multi line object.  The first line contains all the 
      text parameters and the second line is the text string.
\item The minimum size is 2 points (1/72 of an inch).
\item There is no maximum size.
\item The coordinate pair is the origin of the text item.
\item The visibility field is an enumerated type:
\begin{itemize}
	\item INVISIBLE = 0 
	\item VISIBLE = 1
\end{itemize}
\item The show\_name\_value is an enumerated type:
\begin{itemize}
	\item SHOW\_NAME\_VALUE = 0  (show both name and value of an attribute)
	\item SHOW\_VALUE = 1 (show only the value of an attribute)
	\item SHOW\_NAME = 2  (show only the name of an attribute)
\end{itemize}
\item The show\_name\_value field is only valid if the string is an attribute
      (string has to be in the form: name=value to be considered an attribute).
\item The angle of the text can only take on one of the following values: 
      0, 90, 180, 270.  A value of 270 will always generate upright text.
\item The alignment/origin field controls the relative location of the 
      origin.
\item The alignment field can take a value from 0 to 8.

The following diagram shows what the values for the alignment field mean:

\begin{center}
\epsfbox{alignment.eps}
\end{center}
\end{itemize}

Example:\newline 
{\tt T 16900 35800 3 10 1 0 0 0}\newline
{\tt Text string!}

A text object with the origin at (16900, 35800), color index 3, 10 points in
size, visible, attribute flags not valid (not an attribute), origin at lower
left, string: Text string!

\subsection{net}

Valid in: Schematic files ONLY

{\bf type x1 y1 x2 y2 color}

\begin{table}[h]
\begin{tabular}{|l|l|l|} \hline
Field 		& Type/unit 	& Description \\ \hline 
\hline
{\bf type} 	& char 		& N \\ \hline
{\bf x1} 	& int/mils	& First X coordinate \\ \hline
{\bf y1} 	& int/mils	& First Y coordinate \\ \hline
{\bf x2} 	& int/mils	& Second X coordinate \\ \hline
{\bf y2} 	& int/mils	& Second Y coordinate \\ \hline
{\bf color} 	& int		& Color index \\ \hline
\end{tabular}
\end{table}

\begin{itemize}
\item Nets can only appear in schematic files.
\item You cannot have a zero length net (the tools will throw them away).
\end{itemize}

Example:\newline
{\tt N 12700 29400 32900 29400 4}

A net segment from (12700, 29400) to (32900, 29400) with color index 4.

\subsection{bus}

Valid in: Schematic files ONLY

{\bf type x1 y1 x2 y2 color ripperdir}

\begin{table}[h]
\begin{tabular}{|l|l|l|} \hline
Field 		& Type/unit 	& Description \\ \hline 
\hline
{\bf type} 	& char		& U \\ \hline
{\bf x1} 	& int/mils 	& First X coordinate \\ \hline
{\bf y1} 	& int/mils	& First Y coordinate \\ \hline
{\bf x2} 	& int/mils	& Second X coordinate \\ \hline
{\bf y2} 	& int/mils	& Second Y coordinate \\ \hline
{\bf color} 	& int		& Color index \\ \hline
{\bf ripperdir} & int 		& Direction of bus rippers \\ \hline
\end{tabular}
\end{table}

\begin{itemize}
\item The ripperdir field for an brand new bus is 0.
\item The ripperdir field takes on a value of 1 or -1 when a net is connected
      to the bus for the first time.  This value indicates the direction of
      the ripper symbol.  The ripper direction is set to the same value for the
      entire life of the bus object. 
\item Buses can only appear in schematic files.
\item You cannot have a zero length bus (the tools will throw them away).
\end{itemize}

Example:\newline
{\tt U 27300 37400 27300 35300 3 0}

A bus segment from (27300, 37400) to (27300, 35300) with color index 3 and
no nets have been connected to this bus segment..


\subsection{pin}

Valid in: Symbol files ONLY

{\bf type x1 y1 x2 y2 color pintype whichend}

\begin{table}[h]
\begin{tabular}{|l|l|l|} \hline
Field 		& Type/unit 	& Description \\ \hline 
\hline
{\bf type} 	& char		& P \\ \hline
{\bf x1} 	& int/mils	& First X coordinate \\ \hline
{\bf y1} 	& int/mils	& First Y coordinate \\ \hline
{\bf x2} 	& int/mils	& Second X coordinate \\ \hline
{\bf y2} 	& int/mils	& Second Y coordinate \\ \hline
{\bf color} 	& int		& Color index \\ \hline
{\bf pintype} 	& int		& Type of pin \\ \hline
{\bf whichend} 	& int		& Specifies the active end \\ \hline
\end{tabular}
\end{table}

\begin{itemize}
\item The pintype is an enumerated type:
\begin{itemize}
\item NORMAL\_PIN = 0
\item BUS\_PIN = 1 {\bf unused}
\end{itemize}
\item The whichend specifies which end point of the pin is the active 
      connection port.  Only this end point can have other pins or nets 
      connected to it.
\item To make the first end point active, whichend should be 0, else to
      specify the other end, whichend should be 1.
\item Pins can only appear in symbol files.
\item You cannot have a zero length pen (the tools will throw them away).
\end{itemize}

Example:\newline
{\tt P 0 200 200 200 1 0 0}

A pin from (0, 200) to (200, 200) with color index 1, a regular pin, and the
first point being the active connection end.


\subsection{component}

Valid in: Schematic files ONLY

{\bf type x y selectable angle mirror basename}

\begin{table}[h]
\begin{tabular}{|l|l|l|} \hline
Field 		& Type/unit 	& Description \\ \hline 
\hline
{\bf type} 	& char 		& C \\ \hline
{\bf x} 	& int/mils	& Origin X coordinate \\ \hline
{\bf y} 	& int/mils	& Origin Y coordinate \\ \hline
{\bf selectable}& int		& Selectable flag \\ \hline
{\bf angle} 	& int/degrees	& Angle of the component \\ \hline
{\bf mirror} 	& int 		& Mirror around Y axis \\ \hline
{\bf basename}  & string	& The filename of the component \\ \hline
\end{tabular}
\end{table}

\begin{itemize}
\item The selectable field is either 1 for selectable or 0 if not selectable.
\item The angle field can only take on the following values: 0, 90, 180, 270.
\item The angle field can only be positive.
\item The mirror flag is 0 if the component is not mirrored (around the Y axis).
\item The mirror flag is 1 if the component is mirrored (around the Y axis).
\item The just basename is the filename of the component.  This filename is 
      not the full path. 
\end{itemize}

Example:\newline
{\tt C 18600 19900 1 0 0 7400-1.sym}

A component whos origin is at (18600,19900), is selectable, not rotated, not
mirrored, and the basename of the component is 7400-1.sym.


\subsection{font}

Valid in: Special font files ONLY

{\bf type character width flag}

\begin{table}[h]
\begin{tabular}{|l|l|l|} \hline
Field 		& Type/unit 	& Description \\ \hline 
\hline
{\bf type} 	& char 		& F \\ \hline
{\bf character} & char		& The character being defined \\ \hline
{\bf width} 	& int/mils	& Width of the character (mils)  \\ \hline
{\bf flag} 	& int		& Special space flag \\ \hline
\end{tabular}
\end{table}

\begin{itemize}
\item This is a special tag and should ONLY show up in font definition files.
\item If the font character being defined is the space character (32) then 
      flag should be 1, otherwise 0.
\end{itemize}


Example:\newline
{\tt F \_ 11 1}

The above font definition is for the space character.

\section{Colors}

In the gEDA/gaf schematic and symbol file format colors are specified via
an integer index.  The relationship between integer and color is based on
object type.  Each object type typically has one or more colors.  Here is 
a table of color index to object type:


\begin{table}[h]
\begin{tabular}{|l|l|} \hline
Color Index	& Object type \\ \hline
\hline
0 		& BACKGROUND\_COLOR 		\\ \hline 
1 		& PIN\_COLOR 			\\ \hline
2 		& NET\_ENDPOINT\_COLOR 		\\ \hline 
3 		& GRAPHIC\_COLOR		\\ \hline
4 		& NET\_COLOR 			\\ \hline                      
5 		& ATTRIBUTE\_COLOR  		\\ \hline
6 		& LOGIC\_BUBBLE\_COLOR 		\\ \hline 
7 		& GRID\_COLOR 			\\ \hline 
8 		& DETACHED\_ATTRIBUTE\_COLOR    \\ \hline 
9 		& TEXT\_COLOR			\\ \hline 
10 		& BUS\_COLOR 			\\ \hline 
11		& SELECT\_COLOR			\\ \hline
12 		& BOUNDINGBOX\_COLOR		\\ \hline
13 		& ZOOM\_BOX\_COLOR 		\\ \hline 
14 		& STROKE\_COLOR			\\ \hline
15 		& LOCK\_COLOR			\\ \hline
16 		& OUTPUT\_BACKGROUND\_COLOR     \\ \hline 
\end{tabular}
\end{table}

The actual color associated with the color index is defined on a per
tool bases.  Objects are typically assigned their corresponding color
index, but it is permissible (sometimes) to assign other color index
values to different object types.


\section{Attributes}

Attributes are enclosed in \{ \} and can only be text.  Attributes are text
items which take on the form name=value.  If it doesn't have name=value,
it's not an attribute.  Attributes are attached to the previous object.
Here's an example:
       
{\tt 
P 988 500 1300 500 1\newline
\{\newline
T 1000 570 5 8 1 1 0\newline
pinseq=3\newline
T 1000 550 5 8 1 1 0\newline
pinnumber=3\newline
\}\newline
}

        The object is a pin which has an attribute pinnumber=3 and
pinseq=3 (name=value).  You can have multiple text objects (both the T
... and text string are required) in between the \{ \}.  As of 20021103,
you can only attached text items as attributes.  Attaching other object
types as attributes is unsupported.

        You can also have "toplevel" attributes.  These attributes are not
attached to any object, but instead are just text objects that take
on the form name=value.  These attributes are useful when you need to
convey some info about a schematic page or symbol and need the netlister
to have access to this info.


\section{Embedded Components}

	Embedded components are components which have all of their definition
stored within the schematic file.  When a users place a component onto a
schematic page, they have the option of making the component embedded.  Other
than storing all the symbol information inside of the schematic, an embedded
component is just any other component.  Embedded components are defined as:

{\tt
C 18600 21500 1 0 0 EMBEDDED555-1.sym \newline
[ \newline
... \newline
... Embedded primitive objects \newline
... \newline
] \newline
}

In the example above, 555-1.sym is the component.  The EMBEDDED tag and
the [ ] are the distinguishing characteristics of embedded components.
componentname.sym must exist in one of the specified component-libraries
if you want to unembed the component.


\newpage
\section{Document Revision History}

\begin{table}[h]
\begin{tabular}{|l|l|} \hline
November 30th, 2002 & Created fileformats.tex from fileformats.html. \\ \hline
December 1st, 2002 & Continued work on this document. \\ \hline
October 4th, 2003 & Added new file format version flag info. \\ \hline
\end{tabular}
\end{table}

\end{document}



